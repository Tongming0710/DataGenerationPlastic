\documentclass{article}
\usepackage{algpseudocode}
\usepackage{algorithm}

\begin{document}
	\begin{algorithm}
		\caption{NorSand model algorithm}
		\begin{algorithmic}[1]
			\State Assembling the tangent matrix $ D_{ijkl}$
			\State $ \delta\sigma_{ij} = D_{ijkl} \delta\epsilon_{kl}$
			\State $\sigma_{trial}=\sigma+\delta\sigma$ 
			\State $J_2=\frac{1}{6}((\sigma_{11}-\sigma_{22})^2+(\sigma_{11}-\sigma_{33})^2+(\sigma_{33}-\sigma_{22})^2+6(\sigma_{12}^{2}+\sigma_{23}^{2}+\sigma_{33}^{2}))$
			\State $p_{trial} = \sigma_ij\delta_{ij}/3, \ q_{trial}=\sqrt{3J_2}$
			\State Then we going to calculating the yield surface
			\State $M_i=M_{tc}\cdot(1-\frac{\chi N | \psi |}{M_{tc}})$
			\State $ p_{max}=p_i\exp{(\frac{\chi\psi}{M_{tc}})}$
			\State $F=q_{trial}-p_{trial}M_i(1-\ln(\frac{p_{trial}}{p_{i}})), 
			\ F_2=p_{trial}-p_{max}$
		\end{algorithmic}
		
	\end{algorithm}
	
	\begin{algorithm}	
		\begin{algorithmic}[1]
			\caption{Summary of the procedure used to calculate the total stress with regard to the give strain increment}
			\Require Given: the total stress $\sigma$ and the total strain $\epsilon$ at current step, $\Delta \epsilon$ is to be applied to the configuration; Yield function $F$
			\Ensure $F(\sigma) \leq 0$
			\State $\Delta \hat{\sigma} = C_{E}\cdot\Delta\epsilon$ \Comment{Calculate the stress increment assuming elastic behavior}
			\State $\hat{\sigma}=\sigma+\Delta \hat{\sigma}$ \Comment{Calculate the trial stress}
			\State $F(\hat{\sigma})$ \Comment{Calculate the value of the yield function, with $\hat{\sigma}$ as the state of stress,  }
			\If{$F(\hat{\sigma}) \leq 0$}
			\State $\sigma=\hat{\sigma}$ \Comment{The strain increment is elastic. In this case, the trial stress is correct; we return.}
			\State \Return $\sigma$
			\Else
			\If{The previous stress is plastic}
			\State $r=0$ \Comment{$r$ is the portion of incremental strain taken elastically.}
			\Else \Comment{There is a transition from elstic to plastic}
			\State Determine $r$ via $F(\sigma+r\cdot\Delta\epsilon)=0$
			\State $\sigma = C_E \cdot r \Delta \epsilon$
			\State $\Delta \epsilon^p = (1-r)\cdot\Delta\epsilon$ \Comment{$\Delta \epsilon^p$ is the elastoplastic strain increment}
			\State $\sigma\gets \sigma + C_{EP} \cdot \Delta \epsilon^p_{i}$ \Comment{Divide $\Delta \epsilon^p$ into $\Delta \epsilon^p_{0\sim i}$ subincrements and iterate}
			\EndIf
			\EndIf
			\State \Return $\sigma$
		\end{algorithmic}
	\end{algorithm}
	
	
	\begin{algorithm}	
		\begin{algorithmic}[1]
			\caption{Procesures to generate the random loading path via gaussian process and principal rotation}
			\Require Given: The ampltitude for principle strains 
			\Ensure $F(\sigma) \leq 0$
			\State $\Delta \hat{\sigma} = C_{E}\cdot\Delta\epsilon$ \Comment{Calculate the stress increment assuming elastic behavior}
			\State $\hat{\sigma}=\sigma+\Delta \hat{\sigma}$ \Comment{Calculate the trial stress}
			\State $F(\hat{\sigma})$ \Comment{Calculate the value of the yield function, with $\hat{\sigma}$ as the state of stress,  }
			\If{$F(\hat{\sigma}) \leq 0$}
			\State $\sigma=\hat{\sigma}$ \Comment{The strain increment is elastic. In this case, the trial stress is correct; we return.}
			\Else
			\If{The previous stress is plastic}
			\State $r=0$ \Comment{$r$ is the portion of incremental strain taken elastically.}
			\Else \Comment{There is a transition from elstic to plastic}
			\State Determine $r$ via $F(\sigma+r\cdot\Delta\epsilon)=0$
			\State $\sigma = C_E \cdot r \Delta \epsilon$
			\State $\Delta \epsilon^{ep} = (1-r)\cdot\Delta\epsilon$ \Comment{$\Delta \epsilon^{ep}$ is the elastoplastic strain increment}
			\State Call \Call{Algorithm for elastoplastic remapping}{$\Delta \epsilon^{ep}$}
			\EndIf
			\EndIf
		\end{algorithmic}
	\end{algorithm}

	\begin{algorithm}	
	\begin{algorithmic}[1]
		\caption{Elastoplastic return mapping (Convergence of this algorithm needs verification)}
		\Require Given: $\mathrm{d} \epsilon$, $H(\bar{\epsilon^{p}})$, and $f(\sigma, H) = 0$
		\State  $k=1$, $\sigma^{(n+1, k)}=\sigma^{(n)}+D\mathrm{d}\epsilon$ \Comment{Calculate the trial stress as the begining of the return mapping iteration}
		\While{$f(\sigma^{(n+1, k)}, H(\bar{\epsilon^{p}}^{(k)})) > 0$}  \Comment{$f=g$ in Association Flow assumption}
		\State $h=-(\frac{\partial f}{\partial \int |\mathrm{d} \bar{\epsilon^{p}}|}\| \frac{\partial f}{\partial \sigma}\|)^{(k)}$
		\State $\mathrm{d}\epsilon^p = (\frac{\frac{\partial f}{\partial \sigma}D \frac{\partial g}{\partial \sigma}}{h+\frac{\partial f}{\partial \sigma}D\frac{\partial g}{\partial \sigma}})^{(k)}\mathrm{d}\epsilon$
		\State $\mathrm{d}\epsilon^p = \mathrm{d}\lambda \frac{\partial g}{\partial \sigma}^{(k)} $
		\State $\bar{\epsilon^{p}}^{(n+1, k)} = \bar{\epsilon^{p}}^{(n+1, k-1)} + \| \mathrm{d}\epsilon^p \| $
		\State $\sigma^{(n+1, k)}= \sigma^{(n+1, k-1)} - D\mathrm{d}\epsilon^p$
		\State $k++$
		\EndWhile
		
		
	\end{algorithmic}
\end{algorithm}

\end{document}